\chapter{The Screens}
Little Sound Dj has nine screens, laid out in a \begin{math} 5 \times 3 \end{math} screen map.

\section{Screen Map}

\begin{figure}[htbp]
\centering
\begin{tabular}{|c|c|c|c|c|}
	\cline{0-4}
	\multicolumn{2}{|c|}{} & \multicolumn{2}{|c|}{} & \\
	\multicolumn{2}{|c|}{Project} & \multicolumn{2}{|c|}{Synth} & Wave \\
	\multicolumn{2}{|c|}{} & \multicolumn{2}{|c|}{}  &\\
        \cline{0-4}
        & & & & \\
        Song & Chain & Phrase & Instr. & Table \\
        & & & & \\
        \cline{0-4}
        \multicolumn{5}{|c|}{} \\
        \multicolumn{5}{|c|}{Groove} \\
        \multicolumn{5}{|c|}{} \\
        \cline{0-4}
\end{tabular}
\end{figure}

The song, chain and phrase screens are used for composing music. The wave, synth,
instrument and table screens are used for making sounds.
Project screen contains project settings, and groove screen controls sequencer timing.
\footnote{There are also three hidden
screens, not shown on the map: The file, word and help screens. We will get back to these later.}

Most time is usually spent in the middle row, as that is where the composing is done.

Move between the screens using \textsc{select+cursor}. Synth and wave screens are overlaid; switch between them by pressing \textsc{select+right}.

\section{Starting and Stopping}

When pressing \textsc{start} in the song screen, Little Sound Dj plays all four
channels. When pressing \textsc{start} in other screens, Little Sound Dj plays the
channel that is being edited.
To start all four channels from some other screen than the song screen,
press \textsc{select+start}.

\section{Song Screen}

\begin{figure}[hbtp]
\centering
\fbox{ \includegraphics{song} }
\caption{Song Screen}
%\label{fig:song}
\end{figure}

The song screen is the highest level of the sequencer. This is where songs are arranged.
It has four columns, one for each channel. The columns contain lists of chains to be played top-down.

A list of button presses can be found in the help screen.

\includegraphics[width=0.8cm]{tip}TIP!
\begin{itemize}
	\item \textit{Press \textsc{b+a} on an empty step to pull up down-below chains.}
	\item \textit{To quickly jump between rows, mark them with \textsc{b,b,b} then jump with \textsc{b+up/down}.}
\end{itemize}

\section{Chain Screen}
A chain holds a list of phrases to be played back. It can represent for example a melody or a bass line. Each phrase has an optional transpose.

Chains can be reused freely between channels. It is particularly useful to play a chain in both pulse channels.

A list of button presses can be found in the help screen.

\begin{figure}[hbtp]
\centering
\fbox{ \includegraphics{chainexample} }
	\caption{This chain plays phrase 3 transposed by 5 semitones, followed by phrases 4-6 without transpose.}
\end{figure}

\section{Phrase Screen}

\begin{figure}[hbtp]
\centering
\fbox{ \includegraphics{phrase} }
\caption{Phrase Screen}
%\label{fig:phrase}
\end{figure}

The phrase screen is where you enter notes. It has four columns: note, instrument, command and command value.

Phrases are shared between channels; that is, any phrase may be played back on any channel. A phrase might however sound very different depending on which channel it is played back on. As an example, if a phrase plays a melody using a pulse instrument, it will probably sound good in the pulse channels but strange in the other channels.

The note column looks different depending on instrument. Usually it shows note and octave, but instruments that play back samples (\textsc{kit}, \textsc{speech}) show sample names instead. Noise instruments show either a byte value (for 15-bit noise) or note and octave (for 7-bit noise).

The instrument column selects instrument. Press \textsc{select+right} to edit an instrument in the instrument screen.

The command column is used to add commands. Each command has a special functionality. For example, the H command hops to next phrase. Tap \textsc{A,A} on any command for quick-help on what it does.

\includegraphics[width=0.8cm]{tip}TIP!
\begin{itemize}
    \item \textit{To change note without retrig, cut the instrument with \textsc{b+a}.}
    \item \textit{Mute kit note columns by pressing \textsc{a+right} until \textsc{off} appears. (Only works for kits with fewer than 15 samples.)}
\end{itemize}

\section{Instrument Screen}

\begin{figure}[hbtp]
\centering
\fbox{ \includegraphics{instr-pulse} }
\caption{Instrument Screen}
\label{fig:instr2}
\end{figure}

There are five instrument types:

\begin{description}
\item[\textsc{pulse}] Makes pulse waves. Used in pulse channels 1 and 2.
\item[\textsc{wave}] Plays waves from the synth and wave screens. Used in the wave channel.
\item[\textsc{kit}] Plays samples from ROM. Used in the wave channel.
\item[\textsc{noise}] Makes pitched 7-bit or 15-bit noise. Used in the noise channel.
\item[\textsc{speech}] Instrument 40 is for speech and is used in the wave channel.
\end{description}

Change instrument type by pressing \textsc{a+cursor} on the \textsc{type} field.

Instruments don't automatically play in a matching channel. You have to make sure the instrument type matches the channel you are using. For example, to use the kit instrument, first make sure that you are in the wave channel.

\subsection{Instrument Parameters}
\label{general-instrument-parameters}

These parameters are used in most instrument types:

\begin{description}
	\item[\textsc{name}] Tap \textsc{a} to name the instrument. Useful for keeping instruments organized. When selecting instrument in phrase screen, this name is shown in the border.
	\item[\textsc{type}] Instrument type.
	\item[\textsc{length}] Sound length.
	\item[\textsc{output}] Send the sound to left/right/both/none speakers. (Use the headphone output to hear the difference!)
    \item[\textsc{pitch}] Controls the behavior of \textsc{p}, \textsc{l} and \textsc{v} commands. \textsc{A+u/d} switches pitch update speed: \textsc{fast} updates pitch at 360 Hz; \textsc{tick} updates pitch every tick; \textsc{step} is like \textsc{fast} except that \textsc{p} does pitch change instead of pitch bend; \textsc{drum} is like \textsc{fast} with logarithmic fall-off, useful for \textsc{p} kicks. \textsc{a+l/r} changes vibrato shape between downwards triangle, saw and square, and upwards triangle, saw and square.
    \item[\textsc{transp.}] When \textsc{on}, the pitch may be affected by project and chain transposes.
    \item[\textsc{cmd/rate}] Slows down \textsc{c} and \textsc{r} commands.
        Also affects \textsc{p} and \textsc{v} commands when \textsc{pitch} is set to \textsc{tick}.
        0=fastest, F=slowest.
    \item[\textsc{table}] Selects a table to run when playing notes. To edit the table, press \textsc{select+right}. To create a new table, press \textsc{a,a}. To clone the table, press \textsc{select+(b,a)}.
        Changing \textsc{tick} to \textsc{step} makes Little Sound Dj step through the table, advancing one step every time the instrument is triggered.
\end{description}

\includegraphics[width=0.8cm]{tip}TIP!
\begin{itemize}
	\item \textit{In the instrument screen, tap \textsc{A,A} on any parameter for quick-help on what it does.}
\end{itemize}

\subsection{Pulse Instrument Parameters}
\label{detune}

\begin{figure}[htpb]
	\begin{center}
\fbox{ \includegraphics{instr-pulse} }
	\end{center}
	\caption{Pulse Instrument Screen}
	\label{fig:instr-pulse}
\end{figure}

\label{pulse-adsr}
\begin{description}
	\item[\textsc{env.}] Three amplitude envelope control values. For each value, first digit sets amplitude, and second digit sets the speed to rise or fall to the next amplitude. Speed 1 is fastest, F is slowest, 0 means hold. As an example, \textsc{env.} 32/AD/10 creates an envelope with fast attack from amplitude 3 to A, slow decay to amplitude 1, then infinite sustain, as shown in figure~\ref{fig:adsrexample}.
	\item[\textsc{wave}] Wave type.
	\item[\textsc{sweep}] Frequency sweep, useful for bass drum and percussion. The first digit sets time, the second sets pitch increase/decrease. Only works on the first pulse channel. 
\end{description}

\begin{figure}[hbtp]
\centering
\includegraphics[width=7cm,angle=-90]{adsrexample} 
	\caption{Amplitude envelope example. \textsc{Env.}=32/AD/10.}
\label{fig:adsrexample}
\end{figure}

The detune settings create interesting phase effects when the same phrase is played on both pulse channels:

\begin{description}
	\item[\textsc{pu2 tsp.}] Transpose pulse channel 2.
	\item[\textsc{finetune}] Detune pulse channel 1 downwards, channel 2 upwards.
\end{description}

\subsection{Wave Instrument Parameters}

The wave instrument plays back synth sounds generated in the \textsc{synth} screen.

\begin{figure}[hbtp]
	\begin{center}
		\fbox { \includegraphics{instr-wave} }
	\end{center}
	\caption{Wave Instrument Screen}
	%\label{fig:instr-wave}
\end{figure}

\begin{description}
    \item[\textsc{volume}] Set amplitude (0=0\%, 1=25\%, 2=50\%, 3=100\%) and left/right output.
    \item[\textsc{finetune}] Detune the sound.
    \item[\textsc{wave}] Select the wave to play back. Waves are edited in the \textsc{wave} screen. When \textsc{play} is set to some other value than \textsc{manual}, this parameter is replaced by \textsc{synth}.
    \item[\textsc{synth}] Select the synth sound to play back. To edit the synth sound being used, press \textsc{select+up} to go to the \textsc{synth} screen. To use a new synth, tap \textsc{a} twice. To clone the synth, press \textsc{select+(b,a)}.
    \item[\textsc{play}] How to play back synth sounds: \textsc{manual}, \textsc{once}, \textsc{loop}, or \textsc{pingpong}. With \textsc{manual}, only a single wave is played. To play through a synth sound in \textsc{manual} mode, use the \textsc{f} command to step through its waves.
	\item[\textsc{speed}] How fast the synth sound should be played back.
	\item[\textsc{length}] Controls the length of the synth sound.
	\item[\textsc{loop pos}] Sets the loop point of the synth sound.
\end{description}

\subsection{Kit Instrument Parameters}

\begin{figure}[hbtp]
	\begin{center}
	\fbox {	\includegraphics{instr-kit} }
	\end{center}
	\caption{Kit Instrument Screen}
	%\label{fig:instr-kit}
\end{figure}

\begin{description}
	\item[\textsc{kit}] Choose the sample kits to use. The first kit will be used in the left note column in the phrase screen; the second kit will be used in the right note column in the phrase screen.
    \item[\textsc{volume}] Set amplitude (0=0\%, 1=25\%, 2=50\%, 3=100\%) and output (left/right/both/off).
	\item[\textsc{finetune}] Pitch shift.
	\item[\textsc{offset}] Set the start loop point. If \textsc{loop} is \textsc{off}, this value can be used for skipping the initial part of a sound.
	\item[\textsc{len}] Sound length. \textsc{All} plays the sample to its end.
	\item[\textsc{loop}] Loop control. \textsc{Off}=don't loop, \textsc{on}=loop sound and start playing from \textsc{offset}, \textsc{atk}=loop sample and start playing from the beginning.
	\item[\textsc{speed}] Full speed or half speed.
	\item[\textsc{clip}] Selects what to do when the signal overshoots while mixing two kits. \textsc{Hard} is the default: Hard clamp the signal to the allowed 0-F range. \textsc{Soft} attenuates the signal to reduce distortion, giving a tape-like effect. \textsc{Fold} and \textsc{wrap} add digital distortion by folding or wrapping the signal around the 0 and F limits. Press \textsc{a+(left, left)} while \textsc{hard} is selected to mix the kits using raw memory contents.
\end{description}

\includegraphics[width=0.8cm]{tip}TIP!
\begin{itemize}
\item \textit{To replace the default sample kits, use the lsdpatcher program.} \url{https://littlesounddj.com/lsd/latest/lsd-patcher/}
\end{itemize}

\subsection{Noise Instrument Parameters}
\label{noise-instrument-parameters}

\begin{figure}[htpb]
	\begin{center}
		\fbox { \includegraphics{instr-noise} }
	\end{center}
	\caption{Noise Instrument Screen}
	\label{fig:instr-noise}
\end{figure}

\begin{description}
	\item[\textsc{env.}] See description of pulse envelope (\ref{pulse-adsr}).
	\item[\textsc{pitch}] When set to \textsc{free}, pitch changes can randomly mute the sound. \textsc{Safe} avoids random mutes by restarting sound after any pitch change.
\end{description}

\subsection{Speech Instrument}

LSDj has 59 speech sounds called allophones. By combining these sounds, it is possible to create any English word or phrase.
\footnote{A full list of allophones can be found in appendix~\ref{allophone-chapter}.}

The speech instrument is locked to instrument number 40 and can only be used in the wave channel. It contains 14 word slots, by default named W-0 to W-D.

\begin{figure}[htpb]
	\begin{center}
	\fbox{\includegraphics{speech}}
	\end{center}
	\caption{Speech Instrument Screen}
	\label{fig:speech}
\end{figure}
\begin{figure}[htpb]
	\begin{center}
	\fbox{\includegraphics{word}}
	\end{center}
	\caption{Example Word}
	\label{fig:word}
\end{figure}

To edit a word, press \textsc{select+right} to get to the word screen. The left column contains the allophones to be played, the right column sets their duration. The word in figure~\ref{fig:word} is supposed to say ``Little Sound Dj.''

To make the words easy to remember, rename them by tapping A in the speech instrument screen.

\begin{figure}[hbtp]
	\includegraphics[width=0.84cm]{tip}TIP!
	\begin{itemize}
		\item When selecting allophones, care about how they sound, not how they are spelled.
		\item A sound may be different depending on its position within a word. For example, the K in ``coop'' will sound different from the K's in ``keep'' and ``speak''.
	\end{itemize}
\end{figure}

\section{Table Screen}

Tables are sequences of transposes, commands and amplitude changes which can be run at any speed and applied to any channel. By setting a table in the instrument screen, the table will start every time you play the instrument. This allows you to create more interesting sounds than would be possible using the instrument screen alone.

Tables contain six columns. The first column is the envelope column, used to create custom amplitude envelopes. Next is the transpose column, used to transpose the played note by a number of semitones. The other columns are command columns like the one in the phrase screen.

The default table speed of one tick per step can be changed using the G command. To view different tables, press \textsc{b+cursor}.

\includegraphics[width=0.8cm]{tip}TIP!
\begin{itemize}
	\item \textit{Press \textsc{select+right} on an A command in the phrase screen to edit that table. To jump back, press \textsc{select+left}.}
\end{itemize}

\subsection{Envelope Example}

The first digit in the envelope column sets the amplitude; the second digit sets for how many ticks the amplitude lasts.
To create a loop, set the first digit to the step you want to hop to, and the second digit to \textsc{H}.

\begin{figure}[htpb]
	\begin{center}
		\fbox{		\includegraphics{table-amp} }
	\end{center}
	\caption{Table Envelope with Tremolo Effect}
	\label{fig:table-amp}
\end{figure}

The table in figure~\ref{fig:table-amp} creates a tremolo effect.

\subsection{Arpeggio Example}

\begin{figure}[htpb]
	\begin{center}
		\fbox{ \includegraphics{table-arp}}
	\end{center}
	\caption{Table Transpose with Major Arpeggio}
	\label{fig:table-arp}
\end{figure}

A typical use for tables is to make arpeggios, which is a musical term for playing notes fast enough to sound like a chord. The table in figure~\ref{fig:table-arp} would form a major chord. Shorter arpeggios can also be made using the \textsc{c} command (see \ref{command-chord}).

\section{Groove Screen}

Grooves control the speed by which your phrases and tables are played back. When used well, grooves can make your music more lively.

\begin{figure}[htbp]
	\begin{center}
		\fbox{ \includegraphics{groove} }
	\end{center}
	\caption{Groove Screen}
	\label{fig:groove}
\end{figure}

The sequencer is based on a time period called \emph{tick}, which is controlled by song tempo.
Ticks are very short: at 125 BPM, there are 50 ticks per second.
Higher tempo means faster ticks, and the other way around. 
In the groove screen, you can control for how many ticks phrase or table steps should last.
The groove in figure~\ref{fig:groove} would make the sequencer spend 6 ticks on every step.

\begin{figure}[htbp]
	\begin{center}
		\fbox{ \includegraphics{groove-swing} }
	\end{center}
	\caption{Swing Example}
	\label{fig:groove-swing}
\end{figure}

You can also use grooves to create custom rhythms. The groove in figure~\ref{fig:groove-swing} would make even note steps last 8 ticks, and odd note steps last 5 ticks, creating a swing effect. Grooves can also be used to create triplets and other complex rhythms.

Groove 0 is the default groove for all phrases, but it is possible to switch to another groove using the \textsc{g} command.
This command also works in tables.

In the groove screen, select the groove you wish to edit by pressing \textsc{b+cursor}.

\includegraphics[width=0.8cm]{tip}TIP!
\begin{itemize}
	\item \textit{ \textsc{A+up/down} changes the swing percentage, while keeping the total number of ticks -- and thus, the resulting song speed -- constant. (Example: Original value is 6/6 = 50\%. Press \textsc{a+up}. Now the value changes to 7/5 = 58\%!) }
	\item \textit{ Press \textsc{select+down} on \textsc{g} commands to edit that groove. }
\end{itemize}

\section{Synth Screen}

The synth screen features a soft synthesizer that generates sounds to be played back by the wave instruments.
Each synth sound uses 10 waves. Synth sound 0 uses waves 00-0F, synth sound 1 uses waves 10-1F, and so on. The generated synth sounds can be viewed in the wave screen (Section~\ref{wave-screen-section}).

In total, there are 16 synth sounds. Choose which one to edit by pressing \textsc{b+cursor}.

\begin{figure}[htbp]
	\begin{center}
		\fbox{\includegraphics{synth}}
	\end{center}
	\caption{Synth Screen}
	\label{fig:synth}
\end{figure}

\subsection{Fixed Synth Settings}

\begin{description}
	\item[\textsc{signal}] Square, saw tooth, triangle or custom wave. \textsc{W.fx} uses a wave in range F0-FF as signal.
\item[\textsc{filter}] Low-pass, high-pass, band-pass or all-pass.
\item[\textsc{dist}] Distortion mode. \textsc{Clip} truncates the wave to \textsc{limit}, \textsc{fold} mirrors the wave around \textsc{limit}, \textsc{wrap} wraps around vertically.
\item[\textsc{phase}] \label{phase}
Compress the waveform horizontally. See figure \ref{fig:phasing} for examples.
\end{description}

\begin{figure}[hbtp]
	\centering
	\subfloat[Phase example. Original wave.]{
		\fbox{\includegraphics{wave}}
	}
	\qquad
	\subfloat[\textsc{Pinch} phasing.]{
	\fbox{\includegraphics{phase-pinch}}
	}

	\subfloat[\textsc{Warp} phasing.]{
		\fbox{\includegraphics{phase-warp}}
	}
	\qquad
	\subfloat[\textsc{Resync} phasing. \textsc{Resyn2} is same, but without anti-aliasing.]{
		\fbox{\includegraphics{phase-resync}}
	}
	\caption{Phase Examples}
	\label{fig:phasing}
\end{figure}

\subsection{Variable Synth Settings}

These settings control the first and last wave of the sound, with a smooth in-between fade.

\begin{description}
\item[\textsc{volume}] Signal volume.
\item[\textsc{cutoff}] Filter cutoff frequency.
\item[\textsc{q}] Resonance control. Boosts the signal around the cutoff frequency, to change how bright or dull the wave sounds.
\item[\textsc{vshift}] Shifts the signal vertically. See figure \ref{fig:vshift} for examples.
\item[\textsc{limit}] Limits the signal vertically using the \textsc{dist} mode. 0-F lowers the volume, 10-FF allows wrapping and adds interesting overtones to loud signals.
\item[\textsc{phase}] Compresses the signal horizontally. See figure \ref{fig:phasing} for examples.
\end{description}

\begin{figure}[htpb]
	\centering
	\subfloat[Vshift example. Original wave.]{
		\fbox{\includegraphics{wave}}
	}

	\subfloat[Vshifted signal. Vshift = 40, clip = wrap.]{
	\fbox{\includegraphics{vshift-40}}
	}

	\subfloat[Vshifted signal. Vshift = 80, clip = wrap.]{
		\fbox{\includegraphics{vshift-80}}
	}
	\caption{Vshift Examples}
	\label{fig:vshift}
\end{figure}

\section{Wave Screen}
\label{wave-screen-section}

In the wave screen, you can view and edit the individual waveforms of the synth sounds. There are 16 synth sounds with 16 waves each. This means that synth sound 0 uses waves 0-F, synth sound 1 uses waves 10-1F, and so on.

A list of button presses can be found in the help screen.

\section{Project Screen}

\begin{figure}[htpb]
	\begin{center}
		\fbox{		\includegraphics{project}}
	\end{center}
	\caption{Project Screen}
	\label{fig:project}
\end{figure}

The project screen (figure~\ref{fig:project}) contains settings that affect the entire program.

\begin{description}
	\item[\textsc{tempo}] Song tempo in BPM. It is possible to change the tempo either by pressing
\textsc{a+cursor}, or by tapping the \textsc{a} button in pace with the desired tempo. When being a
follower in sync mode, you can nudge the tempo by pressing \textsc{a+left/right}, something which
can be useful if devices have drifted out of sync.
	\item[\textsc{transpose}] Adjust the pitch of the pulse and wave instruments by the given number of semitones.
	\item[\textsc{sync}] Connects to other devices using the link port. Read all about sync settings in chapter \ref{sync-chapter}!

	\item[\textsc{clone}] Deep or slim chain cloning. Deep chain cloning will clone a chain's phrases, whereas slim cloning will re-use the old phrases. Read all about cloning in section \ref{cloning}!
	\item[\textsc{look}] Change the font and color set.
	\item[\textsc{key delay/repeat}] Set repeat delay and repeat rate of the Game Boy buttons.
	\item[\textsc{prelisten}] Play notes and instruments while entering them.

	\item[\textsc{help}] Enter help screen. The help screen contains a reference for button presses and a command list.
	\item[\textsc{clean song data}] Merge duplicate chains and phrases and clear unused ones. \label{clean-song-data}
	\item[\textsc{clean instr data}] Merge duplicate tables and clear unused instruments, tables, synths and waves.
	\item[\textsc{load/save song}] Enter file screen. \footnote{The file screen is only available for cartridges that have 1 Mbit SRAM or more. In case your cartridge doesn't have 1 Mbit SRAM, this button will be replaced with a \textsc{reset memory} button.}
\end{description}

The project screen also has two clocks.
The \textsc{worked} clock displays the time spent making the current song, in hours and minutes.
When playing, it is replaced by the \textsc{play} clock, which shows for how long the song has been playing.
The \textsc{total} clock shows how long the cartridge has been used in total, in days, hours and minutes.

\includegraphics[width=0.8cm]{tip}TIP!
\begin{itemize}
\item \textit{To replace the default fonts and palettes, use the lsdpatcher program.} \url{https://littlesounddj.com/lsd/latest/lsd-patcher/}
\end{itemize}

\subsection{Total Memory Reset}
\label{total-memory-reset}

Press \textsc{select+a+b} on \textsc{load/save file} to erase all songs and bring back the cartridge to its default state.
Generally, this is only useful if the cartridge got scrambled.

\section{File Screen}

\begin{figure}[htpb]
	\begin{center}
		\fbox{\includegraphics{file}}
	\end{center}
	\caption{File Screen}
	\label{fig:file}
\end{figure}

The file screen (figure~\ref{fig:file}) is entered by pressing the \textsc{load/save file} button in the project screen. It is used for saving the song you are working on to the storage memory. It can also be used to load songs from the storage memory to the work memory. The file screen allows you to keep up to 32 songs on one cartridge.

The file screen is only available for cartridges that have 64 kb SRAM or more.

\begin{description}
	\item[\textsc{file}] Shows the file name of the song you are working on. The exclamation mark (\textsc{!}) indicates when changes have been made to a song.
	\item[\textsc{load}] Load a song. Press \textsc{a}, select the file to load and press \textsc{a} again.
	\item[\textsc{save}] Save song. Press \textsc{a}, select the slot to save to and enter the file name.
	\item[\textsc{erase}] Erase a song. Press \textsc{a}, select the file to erase and press \textsc{a} again.
	\item[\textsc{blocks used}] Shows how much of the storage memory that is used. One block equals 512 bytes. The digits on the bottom are hexadecimal, meaning there is a total of BF * 512 = 97,792 available bytes.
\end{description}

Press \textsc{b} to return to the project screen.

\begin{figure}[hbtp]
	\includegraphics[width=0.8cm]{tip}TIP!
	\begin{itemize}
		\item \textit{To manage songs, use the lsdpatcher program.} \url{https://littlesounddj.com/lsd/latest/lsd-patcher/}
	\end{itemize}
\end{figure}

\subsection{Song List}

The song list presents song name, version number and file size. When saving, the song is compressed, so the resulting file size will vary with different songs. To start working on a new song, load from the \textsc{(empty)} slot.

\includegraphics[width=0.84cm]{tip}TIP!
\begin{itemize}
    \item{While in the song list, press \textsc{select+a} to load a song without switching to the song screen, and \textsc{start} to start/stop songs. In this way, you can load and play songs without jumping back and forth between screens. This can be handy if you are playing a live show with prepared tracks and want fewer things to think about.}
\end{itemize}

\section{Border Information}

Various useful information is displayed in the screen border.

\begin{figure}[htpb]
	\begin{center}
	\includegraphics[width=11cm]{border}
	\end{center}
	\caption{Border Information}
	\label{fig:border}
\end{figure}

\begin{enumerate}
\item Screen title. Shows what is being edited.
\item The channel being edited, that is, the selected song screen column.
\item Chain position being edited.
\item Current tempo in beats per minute (BPM).
\item Shows what is being played on the channels. \textsc{Mute} appears when pressing \textsc{b+select} or \textsc{b+start}.
\item The waveform being played on the wave channel.
\item The name of the instrument being selected in the phrase screen.
\item Sync status.
\item Screen map.
\end{enumerate}


